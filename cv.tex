\documentclass[11pt,a4paper]{article}

% Packages
\usepackage[margin=2cm]{geometry}
\usepackage{multicol}
\usepackage{tabularx}
\usepackage{enumitem}
\usepackage{fontawesome}
\usepackage{xcolor}
\usepackage{hyperref}
\usepackage{titlesec}
\usepackage{graphicx}

% Colors
\definecolor{primary}{RGB}{70,130,180}
\definecolor{secondary}{RGB}{220,220,220}

% Section formatting
\titleformat{\section}{\Large\bfseries\color{primary}}{\thesection}{1em}{}[\titlerule]
\titlespacing{\section}{0pt}{12pt}{8pt}

\titleformat{\subsection}{\bfseries\color{primary}}{\thesubsection}{1em}{}
\titlespacing{\subsection}{0pt}{8pt}{4pt}


\newcommand{\cvheader}[6]{
    \begin{center}
        \includegraphics[width=0.25\textwidth]{#6} \\[4pt]
            {\Huge\textbf{#1}}\\[4pt]
            {\large #2}\\[2pt]
            \faEnvelope\ #3 \quad \faPhone\ #4 \\[2pt]
            \faLinkedin\ \href{#5}{#5}
    \end{center}
}

\newcommand{\workblock}[4]{
    \begin{tabular}{p{0.25\columnwidth}p{0.65\columnwidth}}
        \textbf{#1} & \textbf{#2} \\
        #3 & #4 \\
    \end{tabular}
}

\begin{document}

\cvheader{Alexander Ignatov}
         {AI}
         {alexander.ignatov.42@gmail.com}
         {+359898842297}
         {https://linkedin.com/in/alexander-ignatov-yalishanda}
         {az.jpg}


\section{Summary}
An AI/ML Engineer/Researcher with over 7 years of experience in Software Engineering.
Aspiring to continue to contribute to the field of Artificial Intelligence and Machine Learning.

\section{Professional Experience}
    \subsection{Full Time Positions}
    \workblock{2024/02 - Present}{AI/ML Engineer}
    {Bronia AI}{
        \begin{itemize}
            \item Co-author a publication (currently in proceedings)
            \item Research and develop ways of detecting acoustic events with the aim of protection
            \item Use advanced sound augmentation techniques to tackle the domain generalization problem
            \item Compare transfer learning to classical learning methods
            \item Analyze in-depth the performance of different models and their hyperparameters
            \item Use Edge TPU optimizations and quantization techniques
            \item Create an MLOps setup for cloud experimentation and deployment (using Azure cloud)
            \item Be part of a growing small startup (team of less than 10 people)
        \end{itemize}
    }

    \workblock{2023/09 - 2023/12}{AI/ML Engineer}
    {Smule}{
        \begin{itemize}
            \item Big Data analysis (trillions of user events)
            \item Various infrastructure tooling used - Snowflake, Luigi, Google Cloud, Docker, Kubernetes, Grafana, Kafka, Redis, Apache Flink, Spark, MLFlow, TeamCity
            \item Researched and worked on improving a state-of-the-art music-related segmentation model (uses lyrics and audio features)
            \item Worked in teams of a few people
        \end{itemize}
    }
    
    \workblock{2022/06 - 2023/09}{iOS Engineer}
    {Smule}{
        \begin{itemize}
            \item Developed new features that directly affect millions of users
            \item Worked woth a custom state-machine-based declarative architecture and designed custom software components
            \item Wrote extensive documentation for new features
            \item Tackled complex issues in a massive mixed codebase (both Obj-C and Swift, both legacy and new components)
            \item Helped develop a Design System package
            \item Helped research and establish new techniques and processes across teams
        \end{itemize}
    }

    \workblock{2019/04 - 2022/06}{Intern to Junior to Mid iOS Engineer}
    {MentorMate}{
        \begin{itemize}
            \item Develop iOS apps in Swift
            \item Develop iOS and Android apps in C\# with Xamarin Native
            \item Maintain iOS apps using Objective-C
            \item Implement different architectural styles and design patterns (MVC, MVVM, Redux) with and without third-party libraries
            \item Use first- and third-party frameworks and libraries: MapKit, CoreData, SwiftUI, Combine, Alamofire, RxSwift, Realm, etc
            \item Work in teams using the Scrum methodology
        \end{itemize}
    }

    \workblock{2018/02 - 2019/04}{Full-Stack Website Developer}
    {IP Consulting}{
        \begin{itemize}
            \item Develop and maintain back-end in PHP
            \item Develop and maintain front-end using jQuery
            \item Maintain software database (MySQL)
            \item Maintain LAMP stack
            \item Participate in the 2018 Taiwan Innovation Expo
            \item Co-inventor of BPO utility model \#4115
        \end{itemize}
    }
    
    \subsection{Part Time Positions}
    \workblock{2021 - Present}{Software Architect}
    {Sathub}{
        \begin{itemize}
            \item Design and implement software components for a EU navy defense project (for government clients) 
            \item Work with geospacial and OSINT data 
            \item Set up software product infrastructure (containerization, databases, servers, etc)
            \item Use Python to create REST API and data processing scripts
            \item Collaborate with leading European manufacturing and tech enterprises
        \end{itemize}
    }
    
    
    \subsection{Part Time Teachings}
    \workblock{2022/10 - Present}{Lecturer}
    {Sofia University, FMI}{
        Teaching and grading students who participate in 
        the elective discipline \textbf{"Python Programming"} 
        that I co-created with 2 colleagues and friends of mine.
        (\href{https://github.com/fmipython}{GitHub repo})
    }

    \workblock{2020/10 - 2021/06}{Teaching Assistant}
    {Sofia University, FMI}{
        Teaching and grading first-grade students in the discipline
        \textbf{"Object-Oriented Programming"}.
    }



\section{Education}

\workblock{2021 - 2023}{M.Sc. in Artificial Intelligence}
{Sofia University, FMI}{
}

\workblock{2017 - 2021}{B.Sc. in Software Engineering}
{Sofia University, FMI}{
}

\section{Publications}
\begin{itemize}[leftmargin=*]
    \item \textbf{Robust Drone Sound Classification Using Transfer
    Learning, Domain Generalization, and 'White
    Canvas' Data Augmentation Technique} (\textit{currently in proceedings}) - A. Ignatov, D. Terziev, T. Todorov. \\
    \textit{Associated with my work at Bronia AI} \\
    Available to read at: \href{https://bronia-ai.github.io/paper-drone-classif-white-canvas/}{https://bronia-ai.github.io/paper-drone-classif-white-canvas}. \\
    A paper that explores the use of transfer learning, domain generalization, and a novel data augmentation technique
    to create a drone sound classification model that improves accuracy and robustness in diverse real-world environments.
    We also compared the embedding spaces of two widely-used audio classification models.
\end{itemize}

\section{Seminars and Webinars}
\begin{itemize}
    \item How LLM can leverage external sources through RAG? (Online, In Bulgarian) \\
    \href{https://softuni.bg/trainings/4485/how-can-llm-leverage-external-sources-through-rag}{https://softuni.bg/trainings/4485/how-can-llm-leverage-external-sources-through-rag} \\
    SoftUni \\
    2024/02/27 \\
    A free webinar at SoftUni on the topic of Retrieval-Augmented Generation, Vector Stores and Large Language Models.

\end{itemize}

\section{Utility Models}
\begin{itemize}[leftmargin=*]
    \item \textbf{Modular Digital Management System for Administrative Activities Operating in a Virtual Private Cloud Environment} - registered in the Bulgarian Patent Office under utility model \href{https://portal.bpo.bg/rd?key=2018004115U}{\#4115}. \\
    \textit{Associated with my work at IP Consulting} \\
    Co-authored a utility model for a modular software management system and participated in the 2018 Taiwan Innovation Expo.
\end{itemize}

\section{Personal projects}
\begin{itemize}[leftmargin=*]
    \item \textbf{Repos and materials for the Python Programming course} \\
    2022/10 - Present \\
    \href{https://github.com/fmipython}{https://github.com/fmipython} \\
    A collection of interactive materials (in the form of Jupyter Notebooks) and assignments for the Python Programming course
    that I co-created and teach at Sofia University. \\
    Currently working on an automatic containerized grader for students' projects.

    \item \textbf{BDZ Delays} \\
    2023 \\
    \href{https://github.com/yalishanda42/BDZ-Delays}{https://github.com/yalishanda42/BDZ-Delays} \\
    An iOS app I published in the App Store that shows real-time train delays in Bulgaria.
    It uses web scraping to obtain the information.
    Libraries used include \href{https://github.com/pointfreeco/swift-composable-architecture}{The Composable Architecture},
    Realm, as well as SwiftUI. Has unit tests and CI/CD.
    
    \item \textbf{Kaomoji Search} \\
    2022 \\
    \href{https://github.com/yalishanda42/kaomoji-search}{https://github.com/yalishanda42/kaomoji-search} \\
    A \href{https://www.raycast.com/yalishanda/kaomoji-search}{Raycast extension} that allows searching for kaomoji and copying them to the clipboard.
    Written in Typescript in a React-like and RxJS style. \\
    At the time of writing has over 1300 installs.
    
    \item \textbf{Py-polynomial} \\
    2020 \\
    \href{https://github.com/yalishanda42/py-polynomial}{https://github.com/yalishanda42/py-polynomial} \\
    A Python library for convenient single-variable polynomial operations. Published to PyPI. \\
    Has CI/CD that executes lints, tests, generates documentation, uploads new version to PyPI,
    checks PRs for code coverage, PEP-8 and pydocstyle compliance, and performs automated code reviews.
    
    \item \textbf{FuzzyKit} \\
    2021 - 2022 \\
    \href{https://github.com/yalishanda42/fuzzykit}{https://github.com/yalishanda42/fuzzykit} \\
    A Swift library implementing Fuzzy Sets and Fuzzy Logic Theory types and operations. \\
    Has CI/CD that executes builds, tests and generates documentation.

    \item \textbf{Stock Options Research and Backtesting Tool} \\
    2024 - Present \\
    \href{https://github.com/yalishanda42/stonksbot/tree/master/options}{https://github.com/yalishanda42/stonksbot/tree/master/options} \\
    A tool for backtesting stock options strategies, providing insights and analytics. \\
    Contains Jupyter notebooks with research and analysis. \\
    Written in Python, uses Pandas, Numpy, and Matplotlib, and Alpaca API for data.
    
    \item \textbf{Smart Locc} \\
    2021 \\
    \href{https://github.com/yalishanda42/smart-locc}{https://github.com/yalishanda42/smart-locc} \\
    An IoT project I did with 2 other university colleagues. It is a smart door lock using ESP8266/ESP32.
    Features include authorization of keys and sending events to a cloud.

    \item \textbf{Scala(ble) Recommender System PoC} \\
    2022 \\
    \href{https://github.com/yalishanda42/scala-recsys}{https://github.com/yalishanda42/scala-recsys} \\
    A proof-of-concept of a machine learning recommender system written in Scala that aims to incorporate
    best of the Object-Oriented and Functional paradigms. 

    \item \textbf{Mynathon} \\
    2020 \\
    \href{https://github.com/yalishanda42/mynathon}{https://github.com/yalishanda42/mynathon} \\
    A programming language that is parsed and transpiled to Python. Serves as a joke.

    \item \textbf{Landbot} \\
    2020, 2023 \\
    \href{https://github.com/yalishanda42/landbot}{https://github.com/yalishanda42/landbot} \\
    A Discord bot that provides simple commands (like songs and rhymes searching) for the Landcore community. 
    Was hosted on Heroku originally, then migrated to Google Cloud (GCP).

    \item \textbf{Assembly File Encrypter/Decrypter} \\
    2019 \\
    \href{https://github.com/yalishanda42/KAPX-praktikum/blob/master/PROJECT.ASM}{https://github.com/yalishanda42/KAPX-praktikum/blob/master/PROJECT.ASM} \\
    A 16-bit DOS Assembly program with a GUI that encrypts and decrypts files using a simple cipher.

\end{itemize}

\end{document}